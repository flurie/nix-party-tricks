% Created 2022-07-10 Sun 16:49
\documentclass[bigger]{beamer}
\usepackage[utf8]{inputenc}
\usepackage[T1]{fontenc}
\usepackage{graphicx}
\usepackage{longtable}
\usepackage{wrapfig}
\usepackage{rotating}
\usepackage[normalem]{ulem}
\usepackage{amsmath}
\usepackage{amssymb}
\usepackage{capt-of}
\usepackage{hyperref}
\usepackage{minted}
\usetheme{metropolis}
\author{Alexander Flurie}
\date{\textit{<2022-07-12 Tue>}}
\title{Nix party tricks}
\hypersetup{
 pdfauthor={Alexander Flurie},
 pdftitle={Nix party tricks},
 pdfkeywords={nix aws ec2 terraform},
 pdfsubject={Some party tricks you can perform with nix and AWS},
 pdfcreator={Emacs 28.1 (Org mode 9.6)}, 
 pdflang={English}}
\begin{document}

\maketitle
\begin{frame}{Outline}
\tableofcontents
\end{frame}


\section{Preamble}
\label{sec:orgcf4ae11}
\begin{frame}[label={sec:orge93ee95}]{Nix is magic}
Surprise, this is a magic show!

Nix is a special kind of magic for specifying pretty much any output you could want.
\end{frame}
\begin{frame}[label={sec:orgb442b5b}]{Nothing up my sleeve}
Behold, a fresh VM.
\end{frame}
\section{\emph{Extremely} abbreviated intro to nix}
\label{sec:orgbdfe28d}
\begin{frame}[label={sec:org543f6c4}]{Nix the First: Language}
Main features:
\begin{itemize}
\item functional
\item dynamic
\item lazy
\item base language is \uline{tiny}
\item Haskell influence (though much divergence since)
\end{itemize}
\end{frame}
\begin{frame}[label={sec:org44d97cf}]{Nix the First: Language (con't)}
Quirky type system:
\begin{itemize}
\item strings have native multiline support
\item URIs
\item paths (relative and absolute)
\item no advanced objects, everything is a set (map)
\item first-class functions
\end{itemize}
\end{frame}
\begin{frame}[label={sec:orgbbc45e5}]{Nix the Second: Package Manager}
nixpkgs
\begin{itemize}
\item Fundamental unit: the derivation
\item Built with and extends Nix language
\item Largest, most active package repository of its kind
\item Many smaller ecosystems, especially by language (2nix)
\end{itemize}
\end{frame}
\begin{frame}[label={sec:org28569cb}]{Nix the Second: Package Manager (con't)}
The Dirty Secret:
\begin{center}
\includegraphics[width=.9\linewidth]{/Users/flurie/.emacs.d/.local/cache/org-persist/7d/4ab595-074b-42d2-beb8-1af9e6a32fc6-7a64e2c687f385d1a58d43451063534f.png}
\end{center}
\end{frame}
\begin{frame}[label={sec:orga8e71c0}]{Nix the Third: Linux Distribution}
NixOS
\begin{itemize}
\item Built on top of nixpkgs and systemd
\item Familiar to users of gentoo and arch
\item Adds in modules for system-level configurability
\end{itemize}
\end{frame}
\begin{frame}[label={sec:org1a2a2e4}]{Nix the Fourth: \emph{misc} tooling}
\begin{itemize}
\item home-manager (nix for \$HOME)
\item nix-darwin (nix for macOS)
\item cachix (arbitrary caching for nix derivations)
\item Hercules CI (CI/CD for nix derivations)
\end{itemize}
\end{frame}
\section{Demo Overview}
\label{sec:org3c81b7a}
\begin{frame}[label={sec:org6f9fc42}]{Purpose}
\begin{itemize}
\item survey of a bunch of common problems and demonstrate solutions with nix
\item whirlwind tour of some great nix ecosystem tooling
\item code is public
\end{itemize}
\end{frame}
\begin{frame}[label={sec:org6a09cd0}]{Let's install nix!}
\begin{itemize}
\item Go to \url{https://nixos.org}
\item select \alert{Download}
\item Follow multi-user installation instructions (unless you're on something weird like WSL)
\end{itemize}
\end{frame}
\section{First party trick: nix for managing development environments}
\label{sec:org872305d}
\begin{frame}[label={sec:orge86b34f}]{Misc tools for environment management}
\begin{itemize}
\item direnv: automate environment switching in shell
\item devshell: manage all your development tools per-project with a simple configuration file
\end{itemize}
\end{frame}
\begin{frame}[label={sec:org9acb822}]{Let's install direnv!}
\begin{itemize}
\item Go to \url{https://direnv.net/\#basic-installation}
\item Follow the NixOS instructions (because I'm not installing Homebrew, boo!) for non-NixOS systems
\item Hook direnv into shell
\end{itemize}
\end{frame}
\begin{frame}[label={sec:org7437e7e},fragile]{Oops, we need git, too}
 We \emph{could} install git the usual way on macOS\ldots{}
(by installing the Xcode command line tools)
\ldots{}but what if we didn't have to?
nixpkgs to the rescue! And this time we don't even need to ``install'' it!
\begin{minted}[]{shell}
nix-shell -p git
\end{minted}
\end{frame}
\begin{frame}[label={sec:orgda8335c},fragile]{Let's grab the code}
 \begin{minted}[]{shell}
git clone
https://github.com/flurie/nix-party-tricks.git
\end{minted}
\end{frame}
\begin{frame}[label={sec:org4c593f7},fragile]{\ldots{}and then let the magic take hold}
 direnv holds a \alert{lot} of power, so be careful with what you allow.

Using nix with direnv provides an additional level of security.

Time to take the ride.

\begin{minted}[]{shell}
direnv allow
\end{minted}
\end{frame}
\begin{frame}[label={sec:org9615ab6}]{Tour our new powers}
\begin{figure}[htbp]
\centering
\includegraphics[width=6cm]{homer.jpg}
\caption{I'm in devshell! I'm in normal shell!}
\end{figure}
\end{frame}
\begin{frame}[label={sec:org4c897dc},fragile]{Enter AWS with train}
 Set the stage for more magic
\begin{minted}[]{shell}
cp -r "$PRJ_ROOT"/support/.aws ~/.aws
\end{minted}
Create some new creds and never have to look at them!
\begin{verbatim}
Log in to AWS

Create new programmatic IAM credentials

Download the csv to our devshell root
\end{verbatim}
\end{frame}
\begin{frame}[label={sec:org222d06c},fragile]{Time to test the thing out}
 \begin{minted}[]{shell}
aws sts get-caller-identity
\end{minted}
\end{frame}
\section{Second party trick: nix for managing ec2s}
\label{sec:orgd235d8e}
\begin{frame}[label={sec:org39a5c10},fragile]{Preamble: terraform to stand up the host}
 \begin{minted}[]{shell}
# $PRJ_ROOT/terraform/ec2
terraform init
terraform apply
\end{minted}
\end{frame}
\begin{frame}[label={sec:org73256d4}]{Misc tools for deployment management}
\begin{itemize}
\item cachix (arbitrary caching for nix derivations)
\item deploy-rs (deploy NixOS to anywhere from anywhere)
\end{itemize}
\end{frame}
\begin{frame}[label={sec:org27fd3e0}]{NixOS on AWS three ways}
\#1: ec2 user data
\end{frame}
\begin{frame}[label={sec:org5e0a5f0},fragile]{NixOS on AWS three ways - \#1}
 \tiny
\begin{minted}[]{terraform}
# main.tf
resource "aws_instance" "nixos" {

  # ...some parts omitted

  root_block_device {
    # need this to be big enough to build things
    volume_size = 20
  }

  user_data = <<END
### https://nixos.org/channels/nixos-22.05 nixos

{ config, pkgs, modulesPath, ... }:
{
  # nix uses same string interpolation as terraform, so we must escape it here
  imports = [ "$${modulesPath}/virtualisation/amazon-image.nix" ];
  ec2.hvm = true;
  system.stateVersion = "22.05";
  environment.systemPackages = with pkgs; [ nix-direnv direnv git ];
  networking.hostName = "nixos-aws";
}
END
}
\end{minted}
\end{frame}
\begin{frame}[label={sec:org0be7cb1},fragile]{NixOS on AWS three ways - \#1}
 We can now enter the machine.

Make sure to use the IP given by terraform.

\begin{minted}[]{shell}
ssh -i /tmp/nixos-ssh.pem root@{IP}
\end{minted}
\end{frame}
\begin{frame}[label={sec:org1d247c2},fragile]{NixOS on AWS three ways - \#1}
 Let's pull down the party tricks repo here\ldots{}
\begin{minted}[]{shell}
git clone
https://github.com/flurie/nix-party-tricks.git
\end{minted}
\end{frame}
\begin{frame}[label={sec:org8af2c8d},fragile]{NixOS on AWS three ways - \#1}
 \ldots{}and activate the devshell!
\begin{minted}[]{shell}
cd nix-party-tricks && direnv allow
\end{minted}
\end{frame}
\begin{frame}[label={sec:orgc1e45f0}]{NixOS on AWS three ways}
\#2: deploy-rs
\end{frame}
\begin{frame}[label={sec:orgb815bfe},fragile]{NixOS on AWS three ways - \#2}
 \tiny
\begin{minted}[]{nix}
deploy = {
  nodes = {
    "aws" = {
      sshUser = "root";
      sshOpts = [ "-i" "/tmp/nixos-ssh.pem" ];
      hostname = "nixos-aws";
      profiles.hello = {
        path = deploy-rs.lib.x86_64-linux.activate.custom
          nixpkgs.legacyPackages.x86_64-linux.hello "./bin/hello";
      };
      profiles.system = {
        path = deploy-rs.lib.x86_64-linux.activate.nixos
          self.nixosConfigurations.aws;
      };
    };
  };
};
\end{minted}
\end{frame}
\begin{frame}[label={sec:orgfefcf4c},fragile]{NixOS on AWS three ways - \#2}
 Deploying to hostname, make sure it's in our /etc/hosts
\begin{minted}[]{shell}
sudo echo "{IP}  nixos-aws" >> /etc/hosts
\end{minted}
\end{frame}
\begin{frame}[label={sec:org6de3be6},fragile]{NixOS on AWS three ways - \#2}
 First deploy: ``hello world''
\begin{minted}[]{shell}
deploy .#aws.hello
\end{minted}
\end{frame}
\begin{frame}[label={sec:org53f8847},fragile]{NixOS on AWS three ways - \#2}
 Second deploy: NixOS system running nginx
\begin{minted}[]{nix}
{
  services.nginx = { enable = true; };
  networking.firewall.allowedTCPPorts = [ 80 ];
}
\end{minted}
Let's deploy!
\begin{minted}[]{shell}
deploy .#aws.system
\end{minted}
\end{frame}
\begin{frame}[label={sec:orgd162f06},fragile]{NixOS on AWS three ways - \#2}
 Now we should get the nginx splash page in a browser
\begin{verbatim}
http://nixos-aws
\end{verbatim}
\end{frame}

\section{Third party trick: nix for managing lambda runtimes}
\label{sec:orgf56d707}
\begin{frame}[label={sec:orgca4d0ee}]{Preamble: more terraform for my lambda}
\end{frame}
\begin{frame}[label={sec:orgad30650}]{Introducing a great nix feature: remote builders}
\end{frame}
\end{document}
