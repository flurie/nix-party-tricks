% Created 2022-07-12 Tue 18:39
\documentclass[bigger]{beamer}
\usepackage[utf8]{inputenc}
\usepackage[T1]{fontenc}
\usepackage{graphicx}
\usepackage{longtable}
\usepackage{wrapfig}
\usepackage{rotating}
\usepackage[normalem]{ulem}
\usepackage{amsmath}
\usepackage{amssymb}
\usepackage{capt-of}
\usepackage{hyperref}
\usepackage{minted}
\usetheme{metropolis}
\author{Alexander Flurie}
\date{\textit{<2022-07-12 Tue>}}
\title{Nix party tricks}
\subtitle{Building EC2s, images, and lambda packages with NixOS}
\institute{Pittsburgh AWS User Group}
\hypersetup{
 pdfauthor={Alexander Flurie},
 pdftitle={Nix party tricks},
 pdfkeywords={nix aws ec2 terraform},
 pdfsubject={Some party tricks you can perform with nix and AWS},
 pdfcreator={Emacs 28.1 (Org mode 9.6)}, 
 pdflang={English}}
\begin{document}

\maketitle
\begin{frame}{Outline}
\tableofcontents
\end{frame}

\metroset{block=fill}

\section{Preamble}
\label{sec:orga6fe86f}
\begin{frame}[label={sec:org4407021}]{Nix is magic}
Surprise, this is a magic show!

Nix is a special kind of magic for specifying pretty much any output you could want.
\end{frame}
\begin{frame}[label={sec:org95c464d}]{Nothing up my sleeve\ldots{}}
Behold, a fresh VM.

I am clicking through these things because I am very serious.
\end{frame}
\begin{frame}[label={sec:org1c5df7e}]{\ldots{}but cards on the table}
There's always something to install when you start.
\end{frame}
\begin{frame}[label={sec:orgd64069a}]{For those playing along at home}
This text is expository, written for flavor and background
\begin{alertblock}{This text highlights an important definition}
Satoshi Nakamoto: last known alias of Charles Ponzi
\end{alertblock}
\begin{exampleblock}{This text is instructional}
Green means go! Do as I say.
\end{exampleblock}
\end{frame}
\section{Demo Overview}
\label{sec:org6c22767}
\begin{frame}[label={sec:org7cc5223}]{Purpose}
\begin{itemize}
\item survey of a bunch of common problems and demonstrate solutions with nix
\item whirlwind tour of some great nix ecosystem tooling
\item code is public: \url{https://github.com/flurie/nix-party-tricks}
\end{itemize}
\end{frame}
\begin{frame}[label={sec:org5138c8c}]{Let's install nix!}
\begin{exampleblock}{Installing nix}
\begin{itemize}
\item Go to \url{https://nixos.org}
\item select \alert{Download}
\item Follow multi-user installation instructions (unless you're on something weird like WSL)
\end{itemize}
\end{exampleblock}
\end{frame}
\begin{frame}[label={sec:org92655b2},fragile]{And let's install some things to make our lives easier (and this demo shorter)}
 \begin{exampleblock}{Rosetta 2}
\texttt{softwareupdate -{}-install-rosetta}
\end{exampleblock}
\begin{exampleblock}{git}
\texttt{nix-env -iA nixpkgs.git}
\end{exampleblock}
\begin{exampleblock}{cachix}
\texttt{nix-env -iA nixpkgs.cachix}
\end{exampleblock}
\begin{exampleblock}{add some stuff to \texttt{/etc/nix/nix.conf}}
\texttt{experimental-features = nix-command flakes}
\texttt{trusted-users = root \$(whoami)}
\end{exampleblock}
\begin{exampleblock}{explicitly add cachix cache}
\texttt{cachix use flurie \&\& sudo pkill nix-daemon}
\end{exampleblock}
\end{frame}
\section{\emph{Extremely} abbreviated intro to nix}
\label{sec:orgbd82ca2}
\begin{frame}[label={sec:org1480c3a}]{Nix the First: Language}
Main features:
\begin{itemize}
\item functional
\item dynamic
\item lazy
\item base language is \uline{tiny}
\item Haskell influence (though much divergence since)
\end{itemize}
\end{frame}
\begin{frame}[label={sec:org1fdf816}]{Nix the First: Language (con't)}
Quirky type system:
\begin{itemize}
\item strings have native multiline support
\item URIs
\item paths (relative and absolute)
\item no advanced objects, everything is a set (map)
\item first-class functions
\end{itemize}
\end{frame}
\begin{frame}[label={sec:org727e4fd}]{Nix the Second: Package Manager}
nixpkgs
\begin{itemize}
\item Fundamental unit: the derivation
\item Built with and extends Nix language
\item Largest, most active package repository of its kind
\item Many smaller ecosystems, especially by language (2nix)
\end{itemize}
\end{frame}
\begin{frame}[label={sec:org5ecbdfb}]{Nix the Second: Package Manager (con't)}
\begin{figure}[htbp]
\centering
\includegraphics[width=.9\linewidth]{/Users/flurie/.emacs.d/.local/cache/org-persist/7d/4ab595-074b-42d2-beb8-1af9e6a32fc6-7a64e2c687f385d1a58d43451063534f.png}
\caption{The Dirty Secret}
\end{figure}
\end{frame}
\begin{frame}[label={sec:orgcd679e0}]{Nix the Third: Linux Distribution}
NixOS
\begin{itemize}
\item Built on top of nixpkgs and systemd
\item Familiar to users of gentoo and arch
\item Adds in modules for system-level configurability
\end{itemize}
\end{frame}
\begin{frame}[label={sec:org3249ba8},fragile]{Nix the Fourth: \emph{misc} tooling}
 \begin{alertblock}{Tools worth knowing}
\begin{itemize}
\item home-manager: nix for \texttt{\$HOME}
\item nix-darwin: nix for macOS
\item cachix: arbitrary caching for nix derivations
\item Hercules CI: CI/CD for nix derivations
\end{itemize}
\end{alertblock}
\end{frame}
\section{First party trick: nix for managing development environments}
\label{sec:org9f49e66}
\begin{frame}[label={sec:org237b2cd}]{Misc tools for environment management}
\begin{alertblock}{Tools we will use in this section}
\begin{itemize}
\item direnv: automate environment switching in shell
\item devshell: manage all your development tools per-project with a simple configuration file
\end{itemize}
\end{alertblock}
\end{frame}
\begin{frame}[label={sec:org542720d}]{Let's install direnv!}
\begin{exampleblock}{Installing direnv}
\begin{itemize}
\item Go to \url{https://direnv.net/\#basic-installation}
\item Follow the NixOS instructions (because I'm not installing Homebrew, boo!) for non-NixOS systems
\item Hook direnv into shell
\end{itemize}
\end{exampleblock}
\end{frame}
\begin{frame}[label={sec:org5b43268},fragile]{Let's grab the code\ldots{}}
 \begin{exampleblock}{Clone me on GitHub}
\begin{minted}[]{shell}
git clone
https://github.com/flurie/nix-party-tricks.git
\end{minted}
\end{exampleblock}
\end{frame}
\begin{frame}[label={sec:org63163e5},fragile]{\ldots{}and then let the magic take hold}
 direnv holds a \alert{lot} of power, so be careful with what you allow.

Using nix with direnv provides an additional level of security.

\begin{exampleblock}{Time to take the ride.}
\begin{minted}[]{shell}
direnv allow
\end{minted}
\end{exampleblock}
\end{frame}
\begin{frame}[label={sec:org2f7b4c8}]{Tour our new powers}
\begin{figure}[htbp]
\centering
\includegraphics[width=6cm]{/Users/flurie/.emacs.d/.local/cache/org-persist/89/fad068-fc50-40bb-9c4b-0aac2d6453f6-4bf5547748a4e01b63da0ea20380add2.jpg}
\caption{I'm in devshell! I'm in normal shell!}
\end{figure}
\end{frame}
\begin{frame}[label={sec:org269cd8a},fragile]{A note about creds}
 \begin{alertblock}{Be safe}
\begin{itemize}
\item \alert{Never} store credentials in a long-lived plaintext config file!
\item use \texttt{credential\_process} to grab creds safely
\end{itemize}
\begin{verbatim}
# ~/.aws/credentials

[default]
credential_process = access_keys_from_csv
\end{verbatim}
\end{alertblock}
\end{frame}
\begin{frame}[label={sec:orgb119207},fragile]{Enter AWS with train}
 \begin{exampleblock}{Set the stage for more magic}
\begin{minted}[]{shell}
cp -r "$PRJ_ROOT"/support/.aws ~/.aws
\end{minted}
\end{exampleblock}
\begin{exampleblock}{You can try this at home, but don't leave the files sitting around.}
\begin{verbatim}
Log in to AWS console

Create new programmatic IAM credentials

Download the csv to our devshell root
\end{verbatim}
\end{exampleblock}
\end{frame}
\begin{frame}[label={sec:org17390c4},fragile]{Time to test the thing out}
 \begin{exampleblock}{Putting it all together}
\begin{minted}[]{shell}
aws sts get-caller-identity
\end{minted}
\end{exampleblock}
\end{frame}
\section{Second party trick: nix for managing ec2s}
\label{sec:org7659037}
\begin{frame}[label={sec:orgaa9e9cd},fragile]{Preamble}
 \begin{exampleblock}{terraform to stand up the host}
\begin{minted}[]{shell}
cd $PRJ_ROOT/terraform/ec2
terraform init
terraform apply
\end{minted}
\end{exampleblock}
\end{frame}
\begin{frame}[label={sec:org71e2303}]{Misc tools for deployment management}
\begin{alertblock}{Tools we will use in this section}
\begin{itemize}
\item cachix: arbitrary caching for nix derivations
\item deploy-rs: deploy NixOS to anywhere from anywhere
\item nixos-generators: generate NixOS machine images of any kind
\end{itemize}
\end{alertblock}
\end{frame}
\begin{frame}[label={sec:orgd300e1d}]{NixOS on AWS three ways}
\#1: ec2 user data
\end{frame}
\begin{frame}[label={sec:org41208f1},fragile]{NixOS on AWS three ways - \#1}
 \tiny
\begin{minted}[]{terraform}
# main.tf
resource "aws_instance" "nixos" {

  # ...some parts omitted

  root_block_device {
    # need this to be big enough to build things
    volume_size = 20
  }

  user_data = <<END
### https://nixos.org/channels/nixos-22.05 nixos

{ config, pkgs, modulesPath, ... }:
{
  # nix uses same string interpolation as terraform, so we must escape it here
  imports = [ "$${modulesPath}/virtualisation/amazon-image.nix" ];
  ec2.hvm = true;
  system.stateVersion = "22.05";
  environment.systemPackages = with pkgs; [ nix-direnv direnv git ];
  networking.hostName = "nixos-aws";
}
END
}
\end{minted}
\end{frame}
\begin{frame}[label={sec:orgcce9f96},fragile]{NixOS on AWS three ways - \#1}
 We can now enter the machine.
\begin{block}{terraform output into ssh config file + hosts file line?}
\end{block}
\begin{exampleblock}{Make sure to use the IP given by terraform.}
\begin{minted}[]{shell}
ssh -i /tmp/nixos-ssh.pem root@{IP}
\end{minted}
\end{exampleblock}
\end{frame}
\begin{frame}[label={sec:org5c31806},fragile]{NixOS on AWS three ways - \#1}
 \begin{exampleblock}{Let's pull down the party tricks repo here as well\ldots{}}
\begin{minted}[]{shell}
git clone
https://github.com/flurie/nix-party-tricks.git
\end{minted}
\end{exampleblock}
\end{frame}
\begin{frame}[label={sec:orge9c6619},fragile]{NixOS on AWS three ways - \#1}
 \begin{exampleblock}{\ldots{}and activate the devshell!}
\begin{minted}[]{shell}
cd nix-party-tricks && direnv allow
\end{minted}

First way done!
\end{exampleblock}
\end{frame}
\begin{frame}[label={sec:orga59ffab}]{NixOS on AWS three ways}
\#2: deploy-rs
\end{frame}
\begin{frame}[label={sec:orgd1e0d6f},fragile]{NixOS on AWS three ways - \#2}
 \tiny
\begin{minted}[]{nix}
deploy = {
  nodes = {
    "aws" = {
      sshUser = "root";
      sshOpts = [ "-i" "/tmp/nixos-ssh.pem" ];
      hostname = "nixos-aws";
      profiles.hello = {
        path = deploy-rs.lib.x86_64-linux.activate.custom
          nixpkgs.legacyPackages.x86_64-linux.hello "./bin/hello";
      };
      profiles.system = {
        path = deploy-rs.lib.x86_64-linux.activate.nixos
          self.nixosConfigurations.aws;
      };
    };
  };
};
\end{minted}
\end{frame}
\begin{frame}[label={sec:org4d0a55a},fragile]{NixOS on AWS three ways - \#2}
 \begin{exampleblock}{let's make sure it's in our /etc/hosts for later}
\begin{minted}[]{shell}
sudo echo "{IP}  nixos-aws" >> /etc/hosts
\end{minted}
\end{exampleblock}
\end{frame}
\begin{frame}[label={sec:org31e01c9},fragile]{NixOS on AWS three ways - \#2}
 \begin{exampleblock}{copy the key over so we can deploy from the machine, then shell in}
\begin{minted}[]{shell}
scp -i /tmp/nixos-ssh.pem /tmp/nixos-ssh.pem root@nixos-aws:~/nix-party-tricks/
ssh -i /tmp/nixos-ssh.pem root@nixos-aws
\end{minted}
\end{exampleblock}
\end{frame}
\begin{frame}[label={sec:org6815946},fragile]{NixOS on AWS three ways - \#2}
 \begin{exampleblock}{First deploy: ``hello world''}
\begin{minted}[]{shell}
# the -s skips the checks, saving us some time
# don't do this at home
cd nix-party-tricks
deploy .#aws.hello -s
\end{minted}
\end{exampleblock}
\end{frame}
\begin{frame}[label={sec:org93d2c66},fragile]{NixOS on AWS three ways - \#2}
 Second deploy: NixOS system running nginx
\begin{minted}[]{nix}
{
  services.nginx = { enable = true; };
  networking.firewall.allowedTCPPorts = [ 80 ];
}
\end{minted}
\begin{exampleblock}{Let's deploy!}
\begin{minted}[]{shell}
deploy .#aws.system -s
\end{minted}
\end{exampleblock}
\end{frame}
\begin{frame}[label={sec:orgd014b45},fragile]{NixOS on AWS three ways - \#2}
 \begin{exampleblock}{Now we should get the nginx splash page in a browser}
\begin{verbatim}
visit http://nixos-aws in a browser
\end{verbatim}

Second way done!
\end{exampleblock}
\end{frame}
\begin{frame}[label={sec:org7f7b947}]{NixOS on AWS three ways}
\#3: nixos-generators
\end{frame}
\begin{frame}[label={sec:orgdf9da33},fragile]{NixOS on AWS three ways \#3}
 \tiny
\begin{minted}[]{nix}
packages.x86_64-linux.awsImage = let system = "x86_64-linux";
      in nixos-generators.nixosGenerate {
        pkgs = nixpkgs.legacyPackages.${system};
        modules = [
          # new hostname for new machine
          networking.hostName = "nixos-aws-ami";
          # mostly stuff you've seen before...
            services.nginx = {
              enable = true;
              virtualHosts.${networking.hostName} = {
                # except now we're serving something special
                root = "${self.packages."${system}".default}/www";
              };
            };
        ];
        format = "amazon";
};
\end{minted}
\end{frame}
\begin{frame}[label={sec:orgdb008cd},fragile]{NixOS on AWS three ways - \#3}
 Let's use our shiny new ec2  for this!

\begin{exampleblock}{But before we do, let's make our user creds available for the sake of simplicity.}
\tiny
\begin{minted}[]{shell}
# from our local
scp -i /tmp/nixos-ssh.pem ./$(whoami)_accessKeys.csv \
    root@nixos-aws:~/nix-party-tricks/

ssh -i /tmp/nixos-ssh.pem root@nixos-aws
\end{minted}
\end{exampleblock}
\end{frame}
\begin{frame}[label={sec:org860397a},fragile]{NixOS on AWS three ways - \#3}
 \begin{exampleblock}{Now let's build the image!}
\begin{minted}[]{shell}
cd $PRJ_ROOT/terraform/ami
nix build .#awsImage
terraform init
terraform apply
\end{minted}

If we're lucky, it will hit the cached version of my image and spare us.

If we're not, I made a trivial change at some point and never cached it, requiring a rebuild.

Declarative build systems are ruthless.
\end{exampleblock}
\end{frame}
\begin{frame}[label={sec:org881f6ac},fragile]{NixOS on AWS three ways - \#3}
 \begin{exampleblock}{Now we should get something special in a browser}
\begin{verbatim}
visit http://nixos-aws-ami in a browser
\end{verbatim}

Third way done!
\begin{center}
\includegraphics[width=.9\linewidth]{/Users/flurie/.emacs.d/.local/cache/org-persist/23/22acb2-8b39-40aa-8418-a65ac13f4a1d-365226c1cab39c066f27d2b0d0f4f68e.jpg}
\end{center}
\end{exampleblock}
\end{frame}
\section{Third party trick: nix for managing lambda runtimes}
\label{sec:org3217961}
\begin{frame}[label={sec:orgf967a50}]{Preamble}
\begin{alertblock}{We will have to manage some stuff by hand.}
Terraform \emph{really} doesn't want to manage container images.
Providers that can make it happen expect to build with docker.
\end{alertblock}
\end{frame}
\begin{frame}[label={sec:orge3f10fb},fragile]{Container image tools}
 \begin{alertblock}{Tools we will use in this section}
\begin{itemize}
\item docker-tools: nixpkgs native OCI-compatible image builder
\item colima: no-fuss container runtimes for macOS and Linux
\end{itemize}
\end{alertblock}
\begin{exampleblock}{Create ECR repo}
\begin{minted}[]{shell}
aws ecr create-repository \
    --repository-name nix
\end{minted}
\end{exampleblock}
\end{frame}
\begin{frame}[label={sec:org16aef04},fragile]{Lambda One}
 \begin{block}{The setup}
\tiny
\begin{minted}[]{nix}
let
  pythonEnv = pkgs.python39.withPackages (ps: with ps; [ awslambdaric ]);
  entrypoint = pkgs.writeScriptBin "entrypoint.sh" ''
    #!${pkgs.bash}/bin/bash
    if [ -z "$AWS_LAMBDA_RUNTIME_API" ]; then
      exec ${pkgs.aws-lambda-rie}/bin/aws-lambda-rie ${pythonEnv}/bin/python3 -m awslambdaric $@
    else
      exec ${pythonEnv}/bin/python3 -m awslambdaric $@
    fi
  '';
  app = pkgs.writeScriptBin "app.py" ''
    #!${pythonEnv}/bin/python3

    import sys

    def handler(event, context):
        return "Hello from AWS Lambda using Python" + sys.version + "!"
  '';
in
# ...
\end{minted}
\end{block}
\end{frame}
\begin{frame}[label={sec:org1a52a58},fragile]{Lambda One (con't)}
 \begin{block}{The image}
\tiny
\begin{minted}[]{nix}
pkgs.dockerTools.buildLayeredImageWithNixDb {
  name = "nix-lambda";
  tag = "latest";
  contents = [ pkgs.bash pkgs.coreutils pythonEnv app pkgs.aws-lambda-rie ];
  config = {
    Entrypoint = [ "${entrypoint}/bin/entrypoint.sh" ];
    Cmd = [ "app.handler" ];
    WorkingDir = "${app}/bin";
  };
}
\end{minted}
\end{block}
\end{frame}
\begin{frame}[label={sec:org749875c},fragile]{Build and push}
 \begin{exampleblock}{Build the image}
\tiny
\begin{minted}[]{shell}
# starting on the build machine
nix build .#lambdaSimple
# all nix builds get a symlink to ./result by default.
# since this is a raw archived OCI image, we can load the path directly.
docker load < result
\end{minted}
\end{exampleblock}
\end{frame}
\begin{frame}[label={sec:orgdb34773},fragile]{Push the image}
 \begin{exampleblock}{Docker login to ECR}
\begin{minted}[]{shell}
aws ecr get-login-password --region us-east-2 | \
    docker login --username AWS --password-stdin \
    "$(aws sts get-caller-identity | jq -r '.Account').ecr.region.amazonaws.com"
\end{minted}
\end{exampleblock}
\begin{exampleblock}{now tag and push to ECR}
\begin{verbatim}
scripts/tag_and_push_lambda.sh
\end{verbatim}
\tiny
\begin{minted}[]{shell}
#! /usr/bin/env nix-shell
#! nix-shell -i bash -p jq
docker tag "$(docker images nix-lambda --format '{{.ID}}')" \
  "$(aws sts get-caller-identity | jq -r '.Account').dkr.ecr.us-east-2.amazonaws.com/nix:latest"
docker push \
  "$(aws sts get-caller-identity | jq -r '.Account').dkr.ecr.us-east-2.amazonaws.com/nix:latest"
\end{minted}
\end{exampleblock}
\end{frame}
\begin{frame}[label={sec:org4e8b1f4},fragile]{Now terraform the rest}
 \begin{exampleblock}{More terraform}
\begin{minted}[]{shell}
cd $PRJ_ROOT/terraform/lambda
terraform init
terraform apply
\end{minted}
\end{exampleblock}
\end{frame}
\begin{frame}[label={sec:org31ebd01},fragile]{See the results}
 \begin{exampleblock}{Calling our function}
\begin{minted}[]{shell}
curl ${function_url}
\end{minted}
\end{exampleblock}
\end{frame}
\begin{frame}[label={sec:orga984a35},fragile]{Lambda Two}
 Let's add some real packages
\begin{block}{The setup}
\tiny
\begin{minted}[]{nix}
  mangum = with pkgs.python39.pkgs;
    buildPythonPackage rec {
      pname = "mangum";
      version = "0.15.0";

      src = fetchPypi {
        inherit pname version;
        sha256 = "sha256-EuhIBhmLI7vVpUubacGu88YhdzRyGbtXyOwRS4prhTc=";
      };

      buildInputs = [ typing-extensions ];

      pythonImportsCheck = [ "mangum" ];

      meta = with pkgs.lib; {
        description = "AWS Lambda support for ASGI applications";
        homepage = "https://github.com/jordaneremieff/mangum";
        license = licenses.mit;
        maintainers = with maintainers; [ ];
      };
    };
  pythonEnv =
    pkgs.python39.withPackages (ps: with ps; [ awslambdaric mangum fastapi ]);
\end{minted}
\end{block}
\end{frame}
\begin{frame}[label={sec:org0f8738a},fragile]{Lambda Two (con't)}
 \begin{block}{The app}
\tiny
\begin{minted}[]{nix}
  app = pkgs.writeScriptBin "app.py" ''
    #!${pythonEnv}/bin/python3

    from fastapi import FastAPI
    from mangum import Mangum

    app = FastAPI()


    @app.get("/")
    def read_root():
        return {"Hello": "World"}


    @app.get("/items/{item_id}")
    def read_item(item_id: int, q: str = None):
        return {"item_id": item_id, "q": q}

    handler = Mangum(app, lifespan="off")
  '';
\end{minted}
\end{block}
\end{frame}
\begin{frame}[label={sec:org1ba2dc7},fragile]{Lambda Two - Up and running}
 \tiny
\begin{minted}[]{shell}
nix build .#lambdaApi
docker load < result
tag_and_push_lambda
\end{minted}
\end{frame}
\begin{frame}[label={sec:org18eca6d}]{And now we cheat}
\begin{exampleblock}{For the sake of simplicity}
Let's just refresh the image in the console.
\end{exampleblock}
\end{frame}
\begin{frame}[label={sec:org789568d},fragile]{Lambda Two - Testing}
 \begin{exampleblock}{Let's try it in a browser}
\begin{itemize}
\item \texttt{/} should get us a hello world
\item \texttt{/docs} should get us the fastapi swagger
\item \texttt{/items/foo} should get us some stuff back
\end{itemize}
\end{exampleblock}
\end{frame}
\begin{frame}[label={sec:org9736064},fragile]{Lambda Three}
 Let's do something with our packages.
The setup is the same, but the app is different.
\begin{block}{The setup}
\tiny
\begin{minted}[]{nix}
  app = pkgs.writeScriptBin "app.py" ''
    #!${pythonEnv}/bin/python3

    from fastapi import FastAPI
    from fastapi.staticfiles import StaticFiles
    from mangum import Mangum

    app = FastAPI()


    @app.get("/")
    def read_root():
        return {"Hello": "World"}

    app.mount("/nixdocs", StaticFiles(directory="${nixPartyTricksDocs}/www", html=True),
        name="nixdocs")

    handler = Mangum(app, lifespan="off")
  '';
# ...
\end{minted}
\end{block}
\end{frame}
\begin{frame}[label={sec:orgfec6870},fragile]{Lambda Two - Up and running}
 \tiny
\begin{minted}[]{shell}
nix build .#lambdaDocs
docker load < result
tag_and_push_lambda
\end{minted}
\end{frame}
\begin{frame}[label={sec:org977cfbd}]{I will repeat and cheat once again}
\begin{exampleblock}{Just do this}
Refresh the image once more.
\end{exampleblock}
\end{frame}
\begin{frame}[label={sec:org4549245},fragile]{Lambda Three - Testing}
 \begin{exampleblock}{Let's try it in a browser}
\begin{itemize}
\item \texttt{/nixdocs/index.html} should have something very interesting for us. I wonder what it could be?
\end{itemize}
\end{exampleblock}
\end{frame}
\begin{frame}[label={sec:org5ab52fa}]{That's it. That's the talk.}
\begin{figure}[htbp]
\centering
\includegraphics[width=2cm]{/Users/flurie/.emacs.d/.local/cache/org-persist/42/345de6-654f-4cb2-a2a5-830f879853f8-8b6c3d72254f6b9c5ac90010c2b71c7c.jpg}
\caption{Any questions?}
\end{figure}
\end{frame}
\end{document}
