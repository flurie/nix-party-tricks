% Created 2022-07-12 Tue 10:27
\documentclass[bigger]{beamer}
\usepackage[utf8]{inputenc}
\usepackage[T1]{fontenc}
\usepackage{graphicx}
\usepackage{longtable}
\usepackage{wrapfig}
\usepackage{rotating}
\usepackage[normalem]{ulem}
\usepackage{amsmath}
\usepackage{amssymb}
\usepackage{capt-of}
\usepackage{hyperref}
\usepackage{minted}
\usetheme{metropolis}
\author{Alexander Flurie}
\date{\textit{<2022-07-12 Tue>}}
\title{Nix party tricks}
\subtitle{Building EC2s, images, and lambda packages with NixOS}
\institute{Pittsburgh AWS User Group}
\hypersetup{
 pdfauthor={Alexander Flurie},
 pdftitle={Nix party tricks},
 pdfkeywords={nix aws ec2 terraform},
 pdfsubject={Some party tricks you can perform with nix and AWS},
 pdfcreator={Emacs 28.1 (Org mode 9.6)}, 
 pdflang={English}}
\begin{document}

\maketitle
\begin{frame}{Outline}
\tableofcontents
\end{frame}

\metroset{block=fill}

\section{Preamble}
\label{sec:org0f6b2dc}
\begin{frame}[label={sec:org15caaa2}]{Nix is magic}
Surprise, this is a magic show!

Nix is a special kind of magic for specifying pretty much any output you could want.
\end{frame}
\begin{frame}[label={sec:org6086051}]{Nothing up my sleeve\ldots{}}
Behold, a fresh VM.

I am clicking through these things because I am very serious.
\end{frame}
\begin{frame}[label={sec:org1e70227}]{\ldots{}but cards on the table}
There's always something to install when you start.
\end{frame}
\begin{frame}[label={sec:org1089d9e}]{For those playing along at home}
This text is expository, written for flavor and background
\begin{alertblock}{This text highlights an important definition}
Satoshi Nakamoto: last known alias of Charles Ponzi
\end{alertblock}
\begin{exampleblock}{This text is instructional}
Green means go! Do as I say.
\end{exampleblock}
\end{frame}
\section{Demo Overview}
\label{sec:orgfc89c11}
\begin{frame}[label={sec:orgcb8c719}]{Purpose}
\begin{itemize}
\item survey of a bunch of common problems and demonstrate solutions with nix
\item whirlwind tour of some great nix ecosystem tooling
\item code is public: \url{https://github.com/flurie/nix-party-tricks}
\end{itemize}
\end{frame}
\begin{frame}[label={sec:org01fd7a2}]{Let's install nix!}
\begin{block}{Installing nix}
\begin{itemize}
\item Go to \url{https://nixos.org}
\item select \alert{Download}
\item Follow multi-user installation instructions (unless you're on something weird like WSL)
\end{itemize}
\end{block}
\end{frame}
\begin{frame}[label={sec:orgcc90896},fragile]{And let's install some things to make our lives easier (and this demo shorter)}
 \begin{block}{Rosetta 2}
\texttt{softwareupdate -{}-install-rosetta}
\end{block}
\begin{block}{git}
\texttt{nix-env -iA nixpkgs.git}
\end{block}
\begin{block}{cachix}
\texttt{nix-env -iA nixpkgs.cachix}
\end{block}
\begin{block}{add some stuff to \texttt{/etc/nix/nix.conf}}
\texttt{experimental-features = nix-command flakes}
\texttt{trusted-users = root \$(whoami)}
\end{block}
\begin{block}{explicitly add cachix cache}
\texttt{cachix use flurie}
\end{block}
\end{frame}
\section{\emph{Extremely} abbreviated intro to nix}
\label{sec:orgca9aad0}
\begin{frame}[label={sec:org515df34}]{Nix the First: Language}
Main features:
\begin{itemize}
\item functional
\item dynamic
\item lazy
\item base language is \uline{tiny}
\item Haskell influence (though much divergence since)
\end{itemize}
\end{frame}
\begin{frame}[label={sec:orgd803ad2}]{Nix the First: Language (con't)}
Quirky type system:
\begin{itemize}
\item strings have native multiline support
\item URIs
\item paths (relative and absolute)
\item no advanced objects, everything is a set (map)
\item first-class functions
\end{itemize}
\end{frame}
\begin{frame}[label={sec:org94595e1}]{Nix the Second: Package Manager}
nixpkgs
\begin{itemize}
\item Fundamental unit: the derivation
\item Built with and extends Nix language
\item Largest, most active package repository of its kind
\item Many smaller ecosystems, especially by language (2nix)
\end{itemize}
\end{frame}
\begin{frame}[label={sec:org9d0e11d}]{Nix the Second: Package Manager (con't)}
\begin{figure}[htbp]
\centering
\includegraphics[width=.9\linewidth]{/Users/flurie/.emacs.d/.local/cache/org-persist/7d/4ab595-074b-42d2-beb8-1af9e6a32fc6-7a64e2c687f385d1a58d43451063534f.png}
\caption{The Dirty Secret}
\end{figure}
\end{frame}
\begin{frame}[label={sec:orgf0ce0dd}]{Nix the Third: Linux Distribution}
NixOS
\begin{itemize}
\item Built on top of nixpkgs and systemd
\item Familiar to users of gentoo and arch
\item Adds in modules for system-level configurability
\end{itemize}
\end{frame}
\begin{frame}[label={sec:org7036745},fragile]{Nix the Fourth: \emph{misc} tooling}
 \begin{alertblock}{Tools worth knowing}
\begin{itemize}
\item home-manager: nix for \texttt{\$HOME}
\item nix-darwin: nix for macOS
\item cachix: arbitrary caching for nix derivations
\item Hercules CI: CI/CD for nix derivations
\end{itemize}
\end{alertblock}
\end{frame}
\section{First party trick: nix for managing development environments}
\label{sec:org6d72be3}
\begin{frame}[label={sec:orge642505}]{Misc tools for environment management}
\begin{alertblock}{Tools we will use in this section}
\begin{itemize}
\item direnv: automate environment switching in shell
\item devshell: manage all your development tools per-project with a simple configuration file
\end{itemize}
\end{alertblock}
\end{frame}
\begin{frame}[label={sec:org96e66ad}]{Let's install direnv!}
\begin{exampleblock}{Installing direnv}
\begin{itemize}
\item Go to \url{https://direnv.net/\#basic-installation}
\item Follow the NixOS instructions (because I'm not installing Homebrew, boo!) for non-NixOS systems
\item Hook direnv into shell
\end{itemize}
\end{exampleblock}
\end{frame}
\begin{frame}[label={sec:orgcc8d6b1},fragile]{Let's grab the code\ldots{}}
 \begin{exampleblock}{Clone me on GitHub}
\begin{minted}[]{shell}
git clone
https://github.com/flurie/nix-party-tricks.git
\end{minted}
\end{exampleblock}
\end{frame}
\begin{frame}[label={sec:org2d01d3a},fragile]{\ldots{}and then let the magic take hold}
 direnv holds a \alert{lot} of power, so be careful with what you allow.

Using nix with direnv provides an additional level of security.

\begin{exampleblock}{Time to take the ride.}
\begin{minted}[]{shell}
direnv allow
\end{minted}
\end{exampleblock}
\end{frame}
\begin{frame}[label={sec:org71a6db5}]{Tour our new powers}
\begin{figure}[htbp]
\centering
\includegraphics[width=6cm]{/Users/flurie/.emacs.d/.local/cache/org-persist/89/fad068-fc50-40bb-9c4b-0aac2d6453f6-4bf5547748a4e01b63da0ea20380add2.jpg}
\caption{I'm in devshell! I'm in normal shell!}
\end{figure}
\end{frame}
\begin{frame}[label={sec:orgc6acc02},fragile]{A note about creds}
 \begin{alertblock}{Be safe}
\begin{itemize}
\item \alert{Never} store credentials in a long-lived plaintext config file!
\item use \texttt{credential\_process} to grab creds safely
\end{itemize}
\begin{verbatim}
# ~/.aws/credentials

[default]
credential_process = access_keys_from_csv
\end{verbatim}
\end{alertblock}
\end{frame}
\begin{frame}[label={sec:org4b05944},fragile]{Enter AWS with train}
 \begin{exampleblock}{Set the stage for more magic}
\begin{minted}[]{shell}
cp -r "$PRJ_ROOT"/support/.aws ~/.aws
\end{minted}
\end{exampleblock}
\begin{exampleblock}{You can try this at home, but don't leave the files sitting around.}
\begin{verbatim}
Log in to AWS console

Create new programmatic IAM credentials

Download the csv to our devshell root
\end{verbatim}
\end{exampleblock}
\end{frame}
\begin{frame}[label={sec:org612ef63},fragile]{Time to test the thing out}
 \begin{exampleblock}{Putting it all together}
\begin{minted}[]{shell}
aws sts get-caller-identity
\end{minted}
\end{exampleblock}
\end{frame}
\section{Second party trick: nix for managing ec2s}
\label{sec:orgc3056ef}
\begin{frame}[label={sec:org1289c55},fragile]{Preamble}
 \begin{exampleblock}{terraform to stand up the host}
\begin{minted}[]{shell}
cd $PRJ_ROOT/terraform/ec2
terraform init
terraform apply
\end{minted}
\end{exampleblock}
\end{frame}
\begin{frame}[label={sec:org24c23c4}]{Misc tools for deployment management}
\begin{alertblock}{Tools we will use in this section}
\begin{itemize}
\item cachix: arbitrary caching for nix derivations
\item deploy-rs: deploy NixOS to anywhere from anywhere
\item nixos-generators: generate NixOS machine images of any kind
\end{itemize}
\end{alertblock}
\end{frame}
\begin{frame}[label={sec:org55618b7}]{NixOS on AWS three ways}
\#1: ec2 user data
\end{frame}
\begin{frame}[label={sec:orgea6880c},fragile]{NixOS on AWS three ways - \#1}
 \tiny
\begin{minted}[]{terraform}
# main.tf
resource "aws_instance" "nixos" {

  # ...some parts omitted

  root_block_device {
    # need this to be big enough to build things
    volume_size = 20
  }

  user_data = <<END
### https://nixos.org/channels/nixos-22.05 nixos

{ config, pkgs, modulesPath, ... }:
{
  # nix uses same string interpolation as terraform, so we must escape it here
  imports = [ "$${modulesPath}/virtualisation/amazon-image.nix" ];
  ec2.hvm = true;
  system.stateVersion = "22.05";
  environment.systemPackages = with pkgs; [ nix-direnv direnv git ];
  networking.hostName = "nixos-aws";
}
END
}
\end{minted}
\end{frame}
\begin{frame}[label={sec:org07932f8},fragile]{NixOS on AWS three ways - \#1}
 We can now enter the machine.
\begin{block}{terraform output into ssh config file + hosts file line?}
\end{block}
\begin{exampleblock}{Make sure to use the IP given by terraform.}
\begin{minted}[]{shell}
ssh -i /tmp/nixos-ssh.pem root@{IP}
\end{minted}
\end{exampleblock}
\end{frame}
\begin{frame}[label={sec:org5fe48e1},fragile]{NixOS on AWS three ways - \#1}
 \begin{exampleblock}{Let's pull down the party tricks repo here as well\ldots{}}
\begin{minted}[]{shell}
git clone
https://github.com/flurie/nix-party-tricks.git
\end{minted}
\end{exampleblock}
\end{frame}
\begin{frame}[label={sec:org77d9ad0},fragile]{NixOS on AWS three ways - \#1}
 \begin{exampleblock}{\ldots{}and activate the devshell!}
\begin{minted}[]{shell}
cd nix-party-tricks && direnv allow
\end{minted}

First way done!
\end{exampleblock}
\end{frame}
\begin{frame}[label={sec:org9368f8f}]{NixOS on AWS three ways}
\#2: deploy-rs
\end{frame}
\begin{frame}[label={sec:orgbc268f9},fragile]{NixOS on AWS three ways - \#2}
 \tiny
\begin{minted}[]{nix}
deploy = {
  nodes = {
    "aws" = {
      sshUser = "root";
      sshOpts = [ "-i" "/tmp/nixos-ssh.pem" ];
      hostname = "nixos-aws";
      profiles.hello = {
        path = deploy-rs.lib.x86_64-linux.activate.custom
          nixpkgs.legacyPackages.x86_64-linux.hello "./bin/hello";
      };
      profiles.system = {
        path = deploy-rs.lib.x86_64-linux.activate.nixos
          self.nixosConfigurations.aws;
      };
    };
  };
};
\end{minted}
\end{frame}
\begin{frame}[label={sec:org2ebceb9},fragile]{NixOS on AWS three ways - \#2}
 \begin{exampleblock}{Deploying to hostname, make sure it's in our /etc/hosts}
\begin{minted}[]{shell}
sudo echo "{IP}  nixos-aws" >> /etc/hosts
\end{minted}
\end{exampleblock}
\end{frame}
\begin{frame}[label={sec:org0d680ff},fragile]{NixOS on AWS three ways - \#2}
 \begin{exampleblock}{First deploy: ``hello world''}
\begin{minted}[]{shell}
# the -s skips the checks, saving us some time
# don't do this at home
deploy .#aws.hello -s
\end{minted}
\end{exampleblock}
\end{frame}
\begin{frame}[label={sec:org6bfa5e4},fragile]{NixOS on AWS three ways - \#2}
 Second deploy: NixOS system running nginx
\begin{minted}[]{nix}
{
  services.nginx = { enable = true; };
  networking.firewall.allowedTCPPorts = [ 80 ];
}
\end{minted}
\begin{exampleblock}{Let's deploy!}
\begin{minted}[]{shell}
deploy .#aws.system -s
\end{minted}
\end{exampleblock}
\end{frame}
\begin{frame}[label={sec:orgbddd3b2},fragile]{NixOS on AWS three ways - \#2}
 \begin{exampleblock}{Now we should get the nginx splash page in a browser}
\begin{verbatim}
visit http://nixos-aws in a browser
\end{verbatim}

Second way done!
\end{exampleblock}
\end{frame}
\begin{frame}[label={sec:org3b60bf2}]{NixOS on AWS three ways}
\#3: nixos-generators
\end{frame}
\begin{frame}[label={sec:org158364c},fragile]{NixOS on AWS three ways \#3}
 \tiny
\begin{minted}[]{nix}
packages.x86_64-linux.awsImage = let system = "x86_64-linux";
      in nixos-generators.nixosGenerate {
        pkgs = nixpkgs.legacyPackages.${system};
        modules = [
          # new hostname for new machine
          networking.hostName = "nixos-aws-ami";
          # mostly stuff you've seen before...
            services.nginx = {
              enable = true;
              virtualHosts.${networking.hostName} = {
                # except now we're serving something special
                root = "${self.packages."${system}".default}/www";
              };
            };
        ];
        format = "amazon";
};
\end{minted}
\end{frame}
\begin{frame}[label={sec:org7143085},fragile]{NixOS on AWS three ways - \#3}
 Let's use our shiny new ec2  for this!

\begin{exampleblock}{But before we do, let's make our user creds available for the sake of simplicity.}
\tiny
\begin{minted}[]{shell}
# from our local
scp -i /tmp/nixos-ssh.pem ./$(whoami)_accessKeys.csv \
    root@nixos-aws:~/nix-party-tricks/

ssh -i /tmp/nixos-ssh.pem root@nixos-aws
\end{minted}
\end{exampleblock}
\end{frame}
\begin{frame}[label={sec:orgdbf9819},fragile]{NixOS on AWS three ways - \#3}
 \begin{exampleblock}{Now let's build the image!}
\begin{minted}[]{shell}
cd $PRJ_ROOT/terraform/ami
terraform init
terraform apply
\end{minted}

If we're lucky, it will hit the cached version of my image and spare us.

If we're not, I made a trivial change at some point and never cached it, requiring a rebuild.

Declarative build systems are ruthless.
\end{exampleblock}
\end{frame}
\begin{frame}[label={sec:org8712c30},fragile]{NixOS on AWS three ways - \#3}
 \begin{exampleblock}{Now we should get something special in a browser}
\begin{verbatim}
visit http://nixos-aws in a browser
\end{verbatim}

Third way done!
\begin{center}
\includegraphics[width=.9\linewidth]{/Users/flurie/.emacs.d/.local/cache/org-persist/23/22acb2-8b39-40aa-8418-a65ac13f4a1d-365226c1cab39c066f27d2b0d0f4f68e.jpg}
\end{center}
\end{exampleblock}
\end{frame}

\section{Third party trick: nix for managing lambda runtimes}
\label{sec:orgc00200c}
\begin{frame}[label={sec:org803d92a}]{Preamble: more terraform for my lambda}
\end{frame}
\begin{frame}[label={sec:org1b34b6d}]{Introducing a great nix feature: remote builders}
\end{frame}
\end{document}
