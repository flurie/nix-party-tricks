% Created 2022-07-09 Sat 16:10
\documentclass[bigger]{beamer}
\usepackage[utf8]{inputenc}
\usepackage[T1]{fontenc}
\usepackage{graphicx}
\usepackage{longtable}
\usepackage{wrapfig}
\usepackage{rotating}
\usepackage[normalem]{ulem}
\usepackage{amsmath}
\usepackage{amssymb}
\usepackage{capt-of}
\usepackage{hyperref}
\usetheme{metropolis}
\author{Alexander Flurie}
\date{\textit{<2022-07-12 Tue>}}
\title{Nix party tricks}
\hypersetup{
 pdfauthor={Alexander Flurie},
 pdftitle={Nix party tricks},
 pdfkeywords={nix aws ec2 terraform},
 pdfsubject={Some party tricks you can perform with nix and AWS},
 pdfcreator={Emacs 28.1 (Org mode 9.6)}, 
 pdflang={English}}
\begin{document}

\maketitle
\begin{frame}{Outline}
\tableofcontents
\end{frame}


\section{Preamble}
\label{sec:orga7d1e72}
\begin{frame}[label={sec:orge0f59e0}]{Nix is magic}
Surprise, this is a magic show!

Nix is a special kind of magic for specifying pretty much any output you could want.
\end{frame}
\begin{frame}[label={sec:org528fef1}]{Nothing up my sleeve}
\end{frame}


\section{\emph{Extremely} abbreviated intro to nix}
\label{sec:orge4e4b5b}
\begin{frame}[label={sec:org92f11a4}]{Nix the First: Language}
Main features:
\begin{itemize}
\item functional
\item dynamic
\item lazy
\item base language is \uline{tiny}
\item Haskell influence (though much divergence since)
\end{itemize}
\end{frame}
\begin{frame}[label={sec:org221555e}]{Nix the First: Language (con't)}
Quirky type system:
\begin{itemize}
\item strings have native multiline support
\item URIs
\item paths (relative and absolute)
\item no advanced objects, everything is a set (map)
\item first-class functions
\end{itemize}
\end{frame}
\begin{frame}[label={sec:orgdba0189}]{Nix the Second: Package Manager}
nixpkgs
\begin{itemize}
\item Fundamental unit: the derivation
\item Built with and extends Nix language
\item Largest, most active package repository of its kind
\item Many smaller ecosystems, especially by language (2nix)
\end{itemize}
\end{frame}
\begin{frame}[label={sec:orgd417cc1}]{Nix the Second: Package Manager (con't)}
The Dirty Secret:
\begin{center}
\includegraphics[width=.9\linewidth]{/Users/flurie/.emacs.d/.local/cache/org-persist/7d/4ab595-074b-42d2-beb8-1af9e6a32fc6-7a64e2c687f385d1a58d43451063534f.png}
\end{center}
\end{frame}
\begin{frame}[label={sec:org5bc54fd}]{Nix the Third: Linux Distribution}
NixOS
\begin{itemize}
\item Built on top of nixpkgs and systemd
\item Familiar to users of gentoo and arch
\item Adds in modules for system-level configurability
\end{itemize}
\end{frame}
\begin{frame}[label={sec:org3a6a84f}]{Nix the Fourth: \emph{misc} tooling}
\begin{itemize}
\item home-manager (nix for \$HOME)
\item nix-darwin (nix for macOS)
\item cachix (arbitrary caching for nix derivations)
\item Hercules CI (CI/CD for nix derivations)
\end{itemize}
\end{frame}
\section{Demo Overview}
\label{sec:org3aedbf8}
\begin{frame}[label={sec:org834285b}]{Purpose}
\begin{itemize}
\item survey of a bunch of common problems and demonstrate solutions with nix
\item whirlwind tour of some great nix ecosystem tooling
\item code is public
\end{itemize}
\end{frame}
\begin{frame}[label={sec:orgb12a4c2}]{Let's install nix!}
\begin{itemize}
\item Go to \url{https://nixos.org}
\item select \alert{Download}
\item Follow multi-user installation instructions (unless you're on something weird like WSL)
\end{itemize}
\end{frame}
\section{First party trick: nix for managing development environments}
\label{sec:orgb6de55d}
\begin{frame}[label={sec:org9626570}]{Misc tools for environment management}
\begin{itemize}
\item direnv: automate environment switching in shell
\item devshell: manage all your development tools per-project with a simple configuration file
\end{itemize}
\end{frame}
\begin{frame}[label={sec:org0ccfe82}]{Let's install direnv!}
\begin{itemize}
\item Go to \url{https://direnv.net/\#basic-installation}
\item Follow the NixOS instructions (because I'm not installing Homebrew, boo!) for non-NixOS systems
\item Hook direnv into shell
\end{itemize}
\end{frame}
\begin{frame}[label={sec:orgf166639},fragile]{Oops, we need git, too}
 We \emph{could} install git the usual way on macOS\ldots{}
(by installing the Xcode command line tools)
\ldots{}but what if we didn't have to?
nixpkgs to the rescue! And this time we don't even need to ``install'' it!
\begin{verbatim}
nix-shell -p git
\end{verbatim}
\end{frame}
\begin{frame}[label={sec:orgeac0dfb},fragile]{Let's grab the code}
 \begin{verbatim}
git clone
https://github.com/flurie/nix-party-tricks.git
\end{verbatim}
\end{frame}
\begin{frame}[label={sec:org0b95c30},fragile]{\ldots{}and then let the magic take hold}
 direnv holds a \alert{lot} of power, so be careful with what you allow.

Using nix with direnv provides an additional level of security.

Time to take the ride.

\begin{verbatim}
direnv allow
\end{verbatim}
\end{frame}
\begin{frame}[label={sec:orgb2d34a4}]{Tour our new powers}
\begin{figure}[htbp]
\centering
\includegraphics[width=6cm]{homer.jpg}
\caption{I'm in devshell! I'm in normal shell!}
\end{figure}
\end{frame}
\begin{frame}[label={sec:orgda50dd8},fragile]{Enter AWS with train}
 Set the stage for more magic
\begin{verbatim}
cp -r "$PRJ_ROOT"/.aws ~/.aws
\end{verbatim}
Create some new creds and never have to look at them!
\begin{verbatim}
Log in to AWS

Create new programmatic IAM credentials

Download the csv to our devshell root
\end{verbatim}
\end{frame}
\begin{frame}[label={sec:orgec9a1fe},fragile]{Time to test the thing out}
 \begin{verbatim}
aws sts get-caller-identity
\end{verbatim}
\end{frame}
\section{Second party trick: nix for managing ec2s}
\label{sec:orga3f615e}
\begin{frame}[label={sec:org53acbcc},fragile]{Preamble: terraform to stand up the host}
 \begin{verbatim}
# $PRJ_ROOT/terraform/ec2
terraform init
terraform apply
\end{verbatim}
\end{frame}
\begin{frame}[label={sec:org18b9a86}]{Misc tools for deployment management}
\begin{itemize}
\item cachix (arbitrary caching for nix derivations)
\item deploy-rs (deploy NixOS to anywhere from anywhere)
\end{itemize}
\end{frame}
\begin{frame}[label={sec:org7f1c6ed},fragile]{NixOS on AWS three ways}
 \#1: AWS instance user data!
\begin{verbatim}
# maim.tf
resource "aws_instance" "nixos" {
  ami                    = data.aws_ami.nixos-latest.id
  instance_type          = "t3.micro"
  key_name               = aws_key_pair.aws_ssh_key.key_name
  vpc_security_group_ids = [aws_security_group.nixos.id]

  root_block_device {
    # need this to be big enough to build things
    volume_size = 20
  }

  tags = {
    Name = "nix-party-tricks"
  }

  user_data = <<END
### https://nixos.org/channels/nixos-22.05 nixos

{ config, pkgs, modulesPath, ... }:
{
  # nix uses same string interpolation as terraform, so we must escape it here
  imports = [ "$${modulesPath}/virtualisation/amazon-image.nix" ];
  ec2.hvm = true;
  system.stateVersion = "22.05";
  environment.systemPackages = with pkgs; [ nix-direnv direnv git ];
  networking.hostName = "nixos-aws";
  nix.extraOptions = "experimental-features = nix-command flakes";
  programs.bash.interactiveShellInit = ''
      eval "$($${pkgs.direnv}/bin/direnv hook bash)"
  '';
}
END
}
\end{verbatim}
\end{frame}
\begin{frame}[label={sec:orgfdc2885}]{NixOS on AWS three ways}
\#2: deploy-rs
\end{frame}
\section{Third party trick: nix for managing lambda runtimes}
\label{sec:orgc4ca418}
\begin{frame}[label={sec:orgc88e507}]{Preamble: more terraform for my lambda}
\end{frame}
\begin{frame}[label={sec:orgc5c9426}]{Introducing a great nix feature: remote builders}
\end{frame}
\end{document}
